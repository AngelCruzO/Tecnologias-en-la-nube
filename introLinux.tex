\documentclass[10pt,a4paper,titlepage]{article}
\usepackage[T1]{fontenc}
\usepackage{graphicx}
\usepackage{amssymb}
\usepackage{hyperref}
\usepackage{float}
\title{Introducción de Linux}
\author{Angel Cruz Olvera}
\begin{document}
	\maketitle
	
	\section*{UNIX}	
	Fue construido en 1969 por un equipo de desarrolladores de los laboratorios Bell en AT\&T Dennis Ritchie, Ken Thompson, Douglas Mcllroy y Joe Osanna. Su nombre original seria UNICS que tiene como significado Uniplexed Information and Computing System. 
	\\
	\\
	Este sistema es de código abierto, lo que el desarrollo y actualización es contribución de los usuarios. Este ademas, es portable, multitarea y multiusuario. UNIX tiene dos componentes principales: la shell y el kernel.
	
	\begin{figure}[H]
		\centering
		\includegraphics[width=0.7\linewidth]{"./images/unix.png"}
		\caption{Logotipo de sistema operativo UNIX}
		\label{fig:imagen-de-whatsapp-2025-08-04-a-las-17}
	\end{figure}
	
	
	\section*{GNU}
	Es un sistema operativo de software libre, el cual consiste en paquetes desarrollado por el proyecto GNU, es decir programas publicados específicamente para el proyecto. Inicio en 1984 por Richard Stallman, su nombre es un acrónimo recursivo de GNU No es UNIX. Posteriormente en 1990, se desarrollo GNU Hurd como kernel propio del proyecto.
	
	\section*{Linux}
	Creado por Linus Torvalds en 1991, siguiendo el concepto de código abierto basado en UNIX. Se compone de varias partes, siendo el kernel el principal de ellos, puesto que es capaz de gestionar los recursos y permite comunicar el hardware y software del equipo.
	
	\subsection*{Arquitectura}
	
	\subsection*{¿Qué es una distribución?}
	
	\subsection*{Comparativa}
	
	\section*{Comandos basicos}
	
	\section*{Administración}
	
	\section*{Networking}
	
	\section*{Referencias}
	FYCGROUP. UNIX: La simplicidad del ingenio, fyccorp.com consultado el 17 de agosto de 2025, recuperado de https://fyccorp.com/unix-la-simplicidad-del-ingenio/	
	\\
	\\
	GNU. ¿Qué es GNU?, www.gnu.org, consultado el 17 de agosto de 2025, recuperado de https://www.gnu.org/home.es.html
	\\
	\\
	Floriano, J.(2024). ¿Qué es el sistema Linux y cuáles son sus ventajas?, BlogSEAS, consultado el 17 de agosto de 2025, recuperado de https://www.seas.es/blog/informatica/que-es-el-sistema-linux-y-cuales-son-sus-ventajas/
	
\end{document}